\section{PID-Regler \formelbuch{147}}

	\subsection{P-Regler - Stationärer Zustand \formelbuch{155}}
		Beim einfachsten linearen Regler, dem P-Typ, besteht ein proportionaler
		Zusammenhang zwischen Fehler $e$ und Stellgrösse $u$.\\
		Der P-Regler reagiert schnell, kann aber den Sprungfehler nicht vollständig
		eliminieren. Er hat einen stationären Fehler.


	\subsection{I-Regler \formelbuch{160}}
		Der reine I-Regler ist allgemein ungünstig, weil er relativ langsam arbeitet
		und die Stabilität schwächt. Ist aber die Regelstrecke nur erster Ordnung
		erziehlt man gute Ergebnisse mit dem I-Regler.\\
		Der I-Regler neigt zum Schwingen.\\
		Bei sprungförmigen Signalen, d.h. für Festwertregelungen hat der I-Regler
		keinen Fehler!


	\subsection{$PT_2$-Glied \formelbuch{163}}
		\begin{tabular}{p{5cm}p{3cm}p{4cm}p{4cm}}
			$T_\omega = 2T_m=\frac{2\pi}{\omega_n \sqrt{1-\zeta^2}}=\frac{2\pi}{\omega}$ & $\omega = \frac{2\pi}{T_\omega}=2\pi f$ &  $T_\omega$: Schwingungsdauer & $\omega_n$: Kennkreisfrequenz\\
			& & $\zeta$:Dämpfungskonstante & $T_m$: Überschwingdauer
		\end{tabular}

		\subsubsection{Dämpfung}
		Optimal bei $\Psi=45$ und $\zeta=\frac{1}{\sqrt{2}}$.
		Dabei erreicht die Regelgrösse $y$ nach $4.3\%$ Überschwingen rasch den	Endwert.
		\subsubsection{Berechnung $\zeta$}
		Aus DGL $\ddot{y}+a_1\dot{y}+a_0 y=\ldots$ folgt $a_1=2\zeta\omega_n$, $a_0=\omega_n^2$ $\Rightarrow \zeta=\frac{a_1}{2\sqrt{a_0}}$ \\
		Mittels Überschwingweite kann $\zeta$ ebenfalls berechnet werden\\
		\begin{tabular}{p{3cm}p{3cm}p{6cm}}
			$\zeta = \frac{1}{\sqrt{1+(\frac{\pi}{c})^2}}$ & $c =ln(\frac{y_m}{y_{\infty}})$ & $y_m$: Überschwingweite
		\end{tabular}

		Weitere Formeln in der LTI-Grundglieder Tabelle

	\subsection{PI-Regler \formelbuch{174}}
		\includegraphics[width=10cm]{./images/PI_Regler.jpg} \\
		\fbox{$G(j\omega)=K_R \frac{1+j\omega T_N}{j\omega T_N}$}\qquad
		\fbox{$arg(G(j\omega))=\arctan(\omega T_N)-\frac{\pi}{2}$}\qquad
		\fbox{$|G(j\omega)| = \frac{K_R \sqrt{1+(\omega T_n)^2}}{T_n \omega}$}

	\subsection{D-Glied \formelbuch{179}}
		Der Differenzierer erzeugt ein Korrektursignal im voraus.
		Nachteilig ist, wenn die Regelgrösse verrauscht ist, dann werden die
		hochfrequenten Störsignale durch die Ableitung verstärkt.\\
		Ein LTI-System, welches ohne D-Glied darstellbar ist, gegebenenfalls durch
		Umformung des Blockdiagramms, heisst realisierbar.  In der Realität wird
		meistens kein reines D-Glied sondern ein $DT_1$-Glied verwendet:\\
		\fbox{$G_{DT_1}(s) = \frac{s T_V}{1+ s T_C}$}

	\subsection{PD-Regler \formelbuch{187} \formelbuch{383}}
		Der PD-Regler entspricht dem inversen PT$_1$-Glied. Meistens wird jedoch
		der $PDT_1$ Regler verwendet.\\
		\fbox{$u=K_R at+K_R T_V a$} 
		\fbox{$G_{PD}(s) = K_R (T_V s + 1) $}
		\fbox{$G_{PDT_1}(s) = K_R \frac{1+s(T_V+T_C)}{1+sT_C}$}

	\subsection{PID-Regler \formelbuch{183} \formelbuch{383}}
		\fbox{$G_{PID}(s) = K_R \left(1 + \frac{1}{s T_N} + s T_V \right)$}
		\fbox{$G_{PIDT_1}(s) = K_R \left(1 + \frac{1}{s T_N} + \frac{s T_V}{1+s T_C} \right)$}

	\subsection{Empirische Einstellregeln \formelbuch{188}}
		\includegraphics[width=13cm]{./images/Empirisch_Regeln.jpg}
		\begin{minipage}[b]{5cm}
        UTF des angenäherten Modells:\\ \\
		$G_0(j\omega)=\frac{K_s}{1+j\omega T_g}e^{-j\omega T_u}$
		\vspace{2.7cm}
		\end{minipage}\\

	\begin{tabular}{|l|p{1.8cm}|l|l|l|l||l|l|}
	    \hline
	    \multicolumn{6}{|c||}{
	      \textbf{Reglereinstellung nach Chien-Hrones-Reswick}
	    } &
	    \multicolumn{2}{|c|}{
	      \textbf{Reglereinstellung nach Ziegler-Nichols}
	    }
		\\ \hline
		\multicolumn{6}{|c||}{
		  $
		  q = \frac{T_g}{T_uK_S} \qquad \mu = \frac{T_g}{T_u}
		  \qquad \text{wenn} \quad \mu
		  \begin{cases}
		    > 10 \rightarrow \text{Strecke gut regelbar} \\
		    < 3 \rightarrow \text{Strecke schlecht regelbar}
		  \end{cases}
		  $
		} & $q=\frac{T_g}{T_uK_s}$ & $K_{R\pi} \qquad T_\pi=\frac{2\pi}{\omega_\pi}$
		\\ \hline
		\textbf{Regler} & \textbf{Regler\-parameter} &
		\multicolumn{2}{|p{3.5cm}|}{\textbf{Führungsverhalten} \newline $y_m$:
		Überschwingen} &
		\multicolumn{2}{|c||}{\textbf{Störverhalten}} &
		\textbf{Sprungantwort} & \textbf{Stabilitätsgrenze}
		\\ \hline
		& & kein $y_m$ & $\frac{y_m}{y_\infty} = 20 \%$ & kein $a$ & $\frac{a}{b}= 20 \%$ & &
		\\ \hline
		P 	& $K_R$ 	& $0.3q$ 	& $0.7q$ 	& $0.3q$ 	& $0.7q$	& $q$ 	& $0.5K_{R\pi}$
		\\ \hline
		PI	& $K_R$		& $0.35q$	& $0.6q$	& $0.6q$	& $0.7q$	& $0.9q$ 	& $0.45K_{R\pi}$
		\\
		    & $T_N$		& $1.17T_g$	& $1T_g$	& $4T_u$	& $2.33T_u$ & $3.33T_u$ &
		    $0.85T_{\pi}$ \\ \hline
		PID & $K_R$		& $0.6q$	& $0.95q$	& $0.95q$	& $1.2q$ 	& $1.2q$ 	& $0.60K_{R\pi}$
		\\
			& $T_N$		& $1T_g$	& $1.36T_g$	& $2.38T_u$	& $2T_u$ 	& $2T_u$	& $0.50T_\pi$
		\\
			& $T_V$		& $0.5T_u$	& $0.47T_u$	& $0.42T_u$	& $0.42T_u$ & $0.5T_u$ 	& $0.125T_\pi$
		\\ \hline
	\end{tabular}


	\subsection{Wind-Up \formelbuch{200}}

