\section{Diverses}

\subsection{Von $G(s)$ zur DGL}
  \begin{multicols}{2}
    \begin{itemize}
      \item $G(s)$ als Bruch aufschreiben.
			\item Zähler = Eingang ($u$) \quad Nenner = Ausgang ($y$)
 			\item $j\omega = \dot{y} \quad (j\omega)^2 = \ddot{y} \quad \ldots $
 			\item Gleiches gillt für den Zähler mit u!
		\end{itemize}
	\columnbreak
		\textbf{Beispiel:} $G(s) = \frac{K(j \omega T + 1)}{(j\omega)^2 T_N + j \omega Tn(1+K_R) + K_R}$ \\			
		\textbf{DGL:} $T_N \ddot{y} + T_N(1+K_R) \dot{y} + K_R y = K(T \dot{u} + u)$
  \end{multicols}    	
    
\subsection{Zeigerdiagramme}
  Soll zu einem Gegebenen Blockschaldbild (oder einer gegebenen Funktion) das Zeigerdiagramm erstellt 
  werden werden die \textbf{Beträge} der Elemente \textbf{multipliziert},
  die \textbf{Winkel} ($arg()$) \textbf{addiert}. Die Werte für die einzelnen Elemente sind der Tabelle in 1. LTI-  
  Grundglieder zu entnehmen. 

\begin{multicols}{2}
  \subsection{Graphisch Phasen-/Verstärkungsreserve}
    \includegraphics[width=7cm]{./images/bode-stabilitaet.png} \\
    Phasen-/Verstärkungsreserve = Phasen-/Amplitudenrand
    
\columnbreak

  \subsection{Störgrössenaufschaltung \formelbuch{204}}
    Bei der Strögrössenaufschaltung wird auch die Hauptstörung (Last) gemessen und verarbeitet.\\
    \textbf{Merkmale:}
    \begin{itemize}[leftmargin=*]
      \item Hauptstörung wird früher erkannt, was die Regelgeschwindigkeit stark verbessert.
      \item Für die Genauigkeit sorgt weiterhin der Regelkreis. Weitere Störungen werden damit auskorrigiert.
      \item Das Führungsverhalten wird druch die Strögrössenaufschaltung nicht beeinflusst. Somit auch nicht
            die Stabilität der Regelung.
      \item Die Zeitverzögerung zwischen Störort und Aufschaltort soll gering sein.
    \end{itemize}
\end{multicols}

\subsection{Kaskadenregelung \formelbuch{205}}
  Die Kaskadenregelung besitzt einen oder mehrere unterlagerte Regelkreise.\\
  \begin{itemize}
    \item Die Hauptstörung, welche oft am Anfang der Strecke angreift, wird rascher erkannt und kann
          schneller auskorrigiert werden.
    \item Die Dynamik wird verbessert.
    \item Die positiven Nebenwirkungen der inneren Regelung, wie Linearisierung der Strecke
          und Reduktion der Parameterempfindlichkeit, sind erwünscht.
    \item Die Kaskadenregelung vereinfacht die Regler und erleichtert ihre Einstellung.
  \end{itemize}

\subsection{diskreter PID \formelbuch{244}}
  Grundaufgaben eines diskreten PID:
  \begin{itemize}
    \item Erfassen der Regelgrösse über die Messeinrichtung (AD-Wandlung)
    \item Fehler bestimmen durch Vergleich
    \item Korrektur berechnen
    \item Korrektur am Prozess über die Stelleinrichtung vornehmen (DA-Wandlung)
  \end{itemize}

\input{sections/BodeApproximation}	