\newgeometry{hmargin = 2cm, vmargin = 2cm}
\thispagestyle{empty}
\begin{landscape}
	\section{LTI-Grundglieder \formelbuch{115 \& 201}}
	\usetikzlibrary{arrows}
	\usetikzlibrary{positioning}
	\usetikzlibrary{decorations.markings}
	\usetikzlibrary{decorations.pathreplacing}
	\tikzset{
		font = \small,
		axis/.style = {
			thick, ->
		},
		locus/.style = {
			ultra thick, hsr-blue,
			postaction = {
				decorate, decoration = {
					markings,
					mark = between positions .2 and .92 step 6mm  with {%
						\draw[ultra thick, ->] (0,0) -- ++(1pt,0);
					},
				},
			},
		},
		dot/.style = {
			circle, draw = hsr-blue, thick,
			fill = white,
			inner sep = 0pt,
			minimum size = 1mm,
		},
		indbrace/.style = {
			gray,
			thick,
			decorate,
			decoration = {
				brace,
				raise = 5pt,
			},
		},
	}
	\centering
	\begin{tabularx}{\linewidth}{%
			c >{\centering}m{4cm}%
			*{2}{ >{\(\displaystyle}c<{\)} }%
			m{3.5cm} m{4cm} X%
	}
		\toprule[2pt]

		\bfseries Typ &
		\bfseries Symbol &
		\textbf{Differentialgleichung} &
		\textbf{Frequenzgang } G &
		\bfseries Nyquistdiagramm &
		\bfseries Bodediagramm &
		\bfseries Anmerkungen \\

		\midrule[1pt]\endhead

		%
		% Proportional
		%
		P &
		\begin{tikzpicture}
			\node[rtprop] (G) {};
			\node[anchor = south west] at (G.north west) {\(K\)};
			\draw[ultra thick, ->] (G.west) ++(-.5,0) node[left] {\(u\)} -- (G.west);
			\draw[ultra thick, ->] (G.east) -- ++(.5,0) node[right] {\(y\)};
		\end{tikzpicture}
		&
		y = Ku & K
		&
		\begin{tikzpicture}
			\draw[axis] (-.2,0) -- (1.8,0) node[right] {Re};
			\draw[axis] (0,-.5) -- (0,.3) node[above] {Im};
			\coordinate (P) at (1.5,0);
			\draw[indbrace] (P) -- node[midway, below = 7pt] {\(K\)} (0,0);
			\node[dot] at (P) {};
		\end{tikzpicture}
		\\

		\midrule[1pt]

		%
		% Delay
		%
		T &
		\begin{tikzpicture}
			\node[rtdelay] (G) {};
			\node[anchor = south east] at (G.north east) {\(T_t\)};
			\draw[ultra thick, ->] (G.west) ++(-.5,0) node[left] {\(u\)} -- (G.west);
			\draw[ultra thick, ->] (G.east) -- ++(.5,0) node[right] {\(y\)};
		\end{tikzpicture}
		&
		y(t) = \varepsilon(t) u(t - T_t) & e^{-j\omega T_t}
		&
		\begin{tikzpicture}
			\draw[axis] (-1,0) -- (1,0) node[right] {Re};
			\draw[axis] (0,-1) -- (0,1) node[above] {Im};
			\draw[locus] (.75,0) arc (360:0:.75);
			\node[dot] at (.75,0) {};
			\node[below right = .5mm] at (.75,0) {\(\omega = 2\pi n\)};
		\end{tikzpicture}
		\\

		\midrule[1pt]

		%
		% PT1
		%
		PT\(_1\) &
		\begin{tikzpicture}
			\node[rtpt1] (G) {};
			\node[anchor = south west] at (G.north west) {\(K\)};
			\node[anchor = south east] at (G.north east) {\(T\)};
			\draw[ultra thick, ->] (G.west) ++(-.5,0) node[left] {\(u\)} -- (G.west);
			\draw[ultra thick, ->] (G.east) -- ++(.5,0) node[right] {\(y\)};
		\end{tikzpicture}
		&
		T\dot{y} + y = Ku & \frac{K}{j\omega T + 1}
		&
		\begin{tikzpicture}
			\draw[axis] (-.2,0) -- (1.8,0) node[right] {Re};
			\draw[axis] (0,-1) -- (0,.6) node[above] {Im};
			\draw[locus] (1.5,0) arc (360:180:.75);
			\draw[indbrace] (0,0) -- node[midway, above = 7pt] {\(K\)} (1.5,0);
			\draw
				(0,0) node[dot] {} % node[above left, anchor = south east] {\(\omega\to\infty\)}
				(1.5,0) node[dot] {} node[above right = 2pt, anchor = south west] {\(\omega = 0\)} 
				(.75,-.75) node[dot] {} node[below] {\(\omega = 1/T\)}
			;
		\end{tikzpicture}
		\\

		\midrule[1pt]

		%
		% PT2
		%
		PT\(_2\) &
		\begin{tikzpicture}
			\node[rtpt2] (G) {};
			\node[anchor = south west] at (G.north west) {\(K\)};
			\node[anchor = south] at (G.north) {\(\zeta\)};
			\node[anchor = south east] at (G.north east) {\(T\)};
			\draw[ultra thick, ->] (G.west) ++(-.5,0) node[left] {\(u\)} -- (G.west);
			\draw[ultra thick, ->] (G.east) -- ++(.5,0) node[right] {\(y\)};
		\end{tikzpicture}
		&
		T^2 \ddot{y} + 2\zeta T \dot{y} + y = Ku &
		\frac{K}{T^2 (j\omega)^2 + 2\zeta T (j\omega) + 1} 
		&
		\begin{tikzpicture}
			\draw[axis] (-.2,0) -- (1.8,0) node[right] {Re};
			\draw[axis] (0,-1) -- (0,.6) node[above] {Im};

			\draw[locus] (1.5,0) 
				to[out = -90, in = 0] (.75,-.8)
				to[out = 180, in = -30] (0, -.7)
				to[out = 160, in = 190] (0,0)
			;

			\draw[indbrace] (0,0) -- node[midway, above = 7pt] {\(K\)} (1.5,0);
			\draw
				(0,0) node[dot] {} % node[above left, anchor = south east] {\(\omega\to\infty\)}
				(1.5,0) node[dot] {} node[above right = 2pt, anchor = south west] {\(\omega = 0\)} 
				(0,-.7) node[dot] {} node[below left = 4pt, anchor = north west] {\(\omega = 1/T\)}
			;
		\end{tikzpicture}
		\\

		\midrule[1pt]

		%
		% Integrator
		%
		I &
		\begin{tikzpicture}
			\node[rtint] (G) {};
			\node[anchor = south west] at (G.north west) {\(K\)};
			\draw[ultra thick, ->] (G.west) ++(-.5,0) node[left] {\(u\)} -- (G.west);
			\draw[ultra thick, ->] (G.east) -- ++(.5,0) node[right] {\(y\)};
		\end{tikzpicture}
		&
		\dot{y} = K u & \frac{K}{j\omega}
		&
		\begin{tikzpicture}
			\draw[axis] (-.2,0) -- (1.8,0) node[right] {Re};
			\draw[axis] (0,-1) -- (0,.3) node[above] {Im};

			\draw[locus] (0,-1) -- (0,0);
			\draw
				(0,0) node[dot] {} node[above right = 2pt] {\(\omega\to\infty\)}
				(0,-.5) node[dot] {} node[left] {\(j\)}
			;
		\end{tikzpicture}
		\\

		\midrule[1pt]

		%
		% Proportional with Integrator
		%
		PI &
		\begin{tikzpicture}
			\node[rtpi] (G) {};
			\node[anchor = south west] at (G.north west) {\(K\)};
			\node[anchor = south east] at (G.north east) {\(T\)};
			\draw[ultra thick, ->] (G.west) ++(-.5,0) node[left] {\(u\)} -- (G.west);
			\draw[ultra thick, ->] (G.east) -- ++(.5,0) node[right] {\(y\)};
		\end{tikzpicture}
		&
		y = K_R \left( u + \int_0^t \frac{u ~d\tau}{T} \right) &
		K \left( 1 + \frac{1}{j\omega T} \right)
		&
		\begin{tikzpicture}
			\draw[axis] (-.2,0) -- (1.8,0) node[right] {Re};
			\draw[axis] (0,-1) -- (0,.4) node[above] {Im};

			\draw[locus] (.7,-1) -- (.7,0);
			\draw[indbrace] (0,0) -- node[midway, above = 7pt] {\(K\)} (.7,0);
			\draw
				(.7,0) node[dot] {} node[above right = 2pt] {\(\omega\to\infty\)}
			;
		\end{tikzpicture}
		\\

		\midrule[1pt]

		%
		% Differentiator
		%
		D &
		\begin{tikzpicture}
			\node[rtdiff] (G) {};
			\node[anchor = south west] at (G.north west) {\(K\)};
			\draw[ultra thick, ->] (G.west) ++(-.5,0) node[left] {\(u\)} -- (G.west);
			\draw[ultra thick, ->] (G.east) -- ++(.5,0) node[right] {\(y\)};
		\end{tikzpicture}
		&
		y = K\dot{u} & j\omega K
		&
		\begin{tikzpicture}
			\draw[axis] (-.2,0) -- (1.8,0) node[right] {Re};
			\draw[axis] (0,-.3) -- (0,1) node[above] {Im};

			\draw[locus] (0,0) -- (0,1);
			\draw
				(0,0) node[dot] {} node[below right = 2pt] {\(\omega = 0\)}
			;
		\end{tikzpicture}
		&
		&
		Nicht realisierbar.
		\\

		\midrule[1pt]

		%
		% Real differentiator
		%
		DT\(_1\) &
		\begin{tikzpicture}
			\node[rtdt1] (G) {};
			\node[anchor = south west] at (G.north west) {\(K\)};
			\node[anchor = south east] at (G.north east) {\(T\)};
			\draw[ultra thick, ->] (G.west) ++(-.5,0) node[left] {\(u\)} -- (G.west);
			\draw[ultra thick, ->] (G.east) -- ++(.5,0) node[right] {\(y\)};
		\end{tikzpicture}
		&
		T\dot{y} + y = K\dot{u} & \frac{j\omega K}{1 + j\omega T}
		&
		\begin{tikzpicture}
			\draw[axis] (-.2,0) -- (1.8,0) node[right] {Re};
			\draw[axis] (0,-.5) -- (0,1.1) node[above] {Im};
			\draw[locus] (0,0) arc (180:0:.75);
			\draw[indbrace] (1.5,0) -- node[midway, below = 7pt] {\(K\)} (0,0);
			\draw
				(0,0) node[dot] {}
				(1.5,0) node[dot] {} node[below right = 2pt, anchor = north west] {\(\omega = 0\)} 
				(.75,.75) node[dot] {} node[above] {\(\omega = 1/T\)}
			;
		\end{tikzpicture}
		\\

		\bottomrule[2pt]
	\end{tabularx}
\end{landscape}
\restoregeometry

\subsection{\"Aquivalente Glieder}

\subsection{PT\(_2\)--Glied}
\subsubsection{D\"ampfung}
\subsubsection{Ersatzmodell}

\subsubsection{Parameter der Sprungantwort}
\renewcommand{\arraystretch}{1.8}
\begin{tabular}{|m{7cm}|m{1cm}m{0.5cm}m{8cm}}
  \cline{1-1}
  $y_{\infty} =A \cdot K$ & &
  $y_{\infty}$: & Endwert\\
  \cline{1-1}  
	$T_\omega = 2T_m=\dfrac{2\pi}{\omega_n \sqrt{1-\zeta^2}}=\dfrac{2\pi}{\omega}$ & &
	$T_{\omega}$: & Schwingungsdauer \\
  \cline{1-1}
	$\varepsilon = \dfrac{\Delta y}{y_{\infty}}$ & &
	\parbox{0.5cm}{
		$\varepsilon$:\\
		$\Delta y$:
	} & 
	\parbox{8cm}{
		Verhältnis von Überschwinger nach $T_e$ zum Endwert\\
		Toleranzbereich der Amplitude nach $T_e$
	}\\
  \cline{1-1}  
	$T_e = \dfrac{\ln\left(\varepsilon\sqrt{1-\zeta^2}\right)}{-\omega_n\cdot\zeta} = 
	\dfrac{1}{\sigma}\ln\left(\dfrac{\varepsilon\omega}{\omega_n}\right)$ & &
	$T_e$: & Einschwingzeit \\
  \cline{1-1}  
	$T_m = \dfrac{\pi}{\omega_n\sqrt{1-\zeta^2}}=\dfrac{\pi}{\omega}$ & &
	$T_m$: & Überschwingungsdauer\\
  \cline{1-1}  
	$y_m = y_{\infty} \cdot e^{\frac{-\pi\cdot\zeta}{\sqrt{1-\zeta^2}}}$ & &
	$y_m$: & Überschwingweite\\
   \cline{1-1}  
	$\omega = \dfrac{1}{T}\sqrt{1-\zeta^2}= \omega_n\sqrt{1-\zeta^2}=\dfrac{2\pi}{T_\omega}=2\pi f$ & &
	$\omega$: & Kreisfrequenz \\
  \cline{1-1}  
	$\omega_n = \dfrac{1}{T}$ & &
	$\omega_n$: & Kennkreisfrequenz \\
  \cline{1-1}  
	$T_a = \frac{\pi - \arccos{(\zeta)}}{\omega_n\cdot\sqrt{1-\zeta^2}}$ & &
	$T_a$: & Anschwingzeit/Anregelzeit \\
  \cline{1-1}  
	$\sigma = -\dfrac{\zeta}{T} = -\zeta\omega_n$ & &
	$\zeta$: & Dämpfungskonstante \\
  \cline{1-1}
	$\delta = \ln{\Big(\frac{y_i}{y_{i+1}}\Big)} = \frac{2\pi \zeta}{\sqrt{1-\zeta^2}}$ & &
	$\delta$: & Logarithmisches Dekrement\\
  \cline{1-1}  
\end{tabular}
\renewcommand{\arraystretch}{1}
	
\subsubsection{Sprungantwort}
\includegraphics[width = 9cm]{./images/pt2StepResp}

\subsubsection{Dämpfung}
Optimal bei $\zeta=\frac{1}{\sqrt{2}}$ ($\Psi=45$).
Dabei erreicht die Regelgrösse $y$ nach $4.3\%$ Überschwingen rasch den	Endwert.

\subsubsection{Berechnung $\zeta$}
\textbf{Aus DGL} $\ddot{y}+a_1\dot{y}+a_0 y=\ldots$ folgt $a_1=2\zeta\omega_n$, 
$a_0=\omega_n^2$
$\Rightarrow \zeta=\frac{a_1}{2\sqrt{a_0}}$ \\
\textbf{Mittels Überschwingweite} kann $\zeta$ ebenfalls berechnet werden\\
\begin{tabular}{p{2.5cm}p{2.5cm}p{4cm}}
$\zeta = \frac{1}{\sqrt{1+(\frac{\pi}{c})^2}}$ & $c =ln(\frac{y_m}{y_{\infty}})$ & $y_m$: Überschwingweite
\end{tabular}

Weitere Formeln in der LTI-Grundglieder Tabelle
