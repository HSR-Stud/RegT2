\documentclass[margin=small]{tex/hsrzf}

\usepackage{tex/hsrstud}
\usepackage{tex/regtec}

\usepackage{polyglossia}
\setdefaultlanguage[variant=swiss]{german}

\usepackage[
    type={CC},
    modifier={by-nc-sa},
    version={4.0},
    lang={german},
]{doclicense}

\usepackage{pdflscape}
\usepackage{tabularx}
\usepackage{booktabs}
\usepackage{multicol}
\usepackage{enumitem}

\usepackage{longtable}

\newcommand\formelbuch[1]{\texttt{\textcolor{red!60!black}{\small S#1}}}

\title{Regelungstechnik 2 -- Formelsammlung}
\author{Braun \& Co., J. Rast, K\"orner, Gwerder, Pross}
% \versioninfo}{$Revision: 2.0 $ - gem\"ass Unterricht Markus Kottmann/FS2013}

\begin{document} 

\maketitle
\tableofcontents

\section*{Lizenz}
\doclicenseThis{}

\newpage

\begin{landscape}
	\thispagestyle{plain}

	\section{LTI-Grundglieder \formelbuch{115 \& 201}}
	\begin{table}[h!]
		\centering
		\begin{longtable}{ c >{\centering}m{3cm} *{2}{ >{\(\displaystyle}c<{\)} } m{4cm} m{4cm} p{5cm}}
			\toprule[2pt]

			\bfseries Typ &
			\bfseries Symbol &
			\textbf{Differentialgleichung} &
			\textbf{Frequenzgang } G &
			\bfseries Nyquistdiagramm &
			\bfseries Bodediagramm &
			\bfseries Anmerkungen \\

			\midrule[1pt]

			P &
			\begin{tikzpicture}
				\node[rtprop] (G) {};
				\node[anchor = south west] at (G.north west) {\(K\)};
				\draw[ultra thick, ->] (G.west) ++(-.5,0) node[left] {\(u\)} -- (G.west);
				\draw[ultra thick, ->] (G.east) -- ++(.5,0) node[right] {\(y\)};
			\end{tikzpicture}
			&
			y = Ku & K
			\\

			\midrule[1pt]

			T &
			\begin{tikzpicture}
				\node[rtdelay] (G) {};
				\node[anchor = south east] at (G.north east) {\(T_t\)};
				\draw[ultra thick, ->] (G.west) ++(-.5,0) node[left] {\(u\)} -- (G.west);
				\draw[ultra thick, ->] (G.east) -- ++(.5,0) node[right] {\(y\)};
			\end{tikzpicture}
			&
			y(t) = \varepsilon(t) u(t - T_t) & e^{-j\omega T_t}
			\\

			\midrule[1pt]

			PT\(_1\) &
			\begin{tikzpicture}
				\node[rtpt1] (G) {};
				\node[anchor = south west] at (G.north west) {\(K\)};
				\node[anchor = south east] at (G.north east) {\(T\)};
				\draw[ultra thick, ->] (G.west) ++(-.5,0) node[left] {\(u\)} -- (G.west);
				\draw[ultra thick, ->] (G.east) -- ++(.5,0) node[right] {\(y\)};
			\end{tikzpicture}
			&
			T\dot{y} + y = Ku & \frac{K}{j\omega T + 1}
			\\

			\midrule[1pt]

			PT\(_2\) &
			\begin{tikzpicture}
				\node[rtpt2] (G) {};
				\node[anchor = south west] at (G.north west) {\(K\)};
				\node[anchor = south] at (G.north) {\(\zeta\)};
				\node[anchor = south east] at (G.north east) {\(T\)};
				\draw[ultra thick, ->] (G.west) ++(-.5,0) node[left] {\(u\)} -- (G.west);
				\draw[ultra thick, ->] (G.east) -- ++(.5,0) node[right] {\(y\)};
			\end{tikzpicture}
			&
			T^2 \ddot{y} + 2\zeta T \dot{y} + y = Ku &
			\frac{K}{T^2 (j\omega)^2 + 2\zeta T (j\omega) + 1} 
			\\

			\midrule[1pt]

			I &
			\begin{tikzpicture}
				\node[rtint] (G) {};
				\node[anchor = south west] at (G.north west) {\(K\)};
				\draw[ultra thick, ->] (G.west) ++(-.5,0) node[left] {\(u\)} -- (G.west);
				\draw[ultra thick, ->] (G.east) -- ++(.5,0) node[right] {\(y\)};
			\end{tikzpicture}
			&
			\dot{y} = K u & \frac{K}{j\omega}
			\\

			\midrule[1pt]

			PI &
			\begin{tikzpicture}
				\node[rtpi] (G) {};
				\node[anchor = south west] at (G.north west) {\(K\)};
				\node[anchor = south east] at (G.north east) {\(T\)};
				\draw[ultra thick, ->] (G.west) ++(-.5,0) node[left] {\(u\)} -- (G.west);
				\draw[ultra thick, ->] (G.east) -- ++(.5,0) node[right] {\(y\)};
			\end{tikzpicture}
			&
			y = K_R \left( u + \int_0^t \frac{u ~d\tau}{T} \right) &
			K \left( 1 + \frac{1}{j\omega T} \right)
			\\

			\midrule[1pt]

			D &
			\begin{tikzpicture}
				\node[rtdiff] (G) {};
				\node[anchor = south west] at (G.north west) {\(K\)};
				\draw[ultra thick, ->] (G.west) ++(-.5,0) node[left] {\(u\)} -- (G.west);
				\draw[ultra thick, ->] (G.east) -- ++(.5,0) node[right] {\(y\)};
			\end{tikzpicture}
			&
			y = K\dot{u} & j\omega K
			\\

			\midrule[1pt]

			DT\(_1\) &
			\begin{tikzpicture}
				\node[rtdt1] (G) {};
				\node[anchor = south west] at (G.north west) {\(K\)};
				\node[anchor = south east] at (G.north east) {\(T\)};
				\draw[ultra thick, ->] (G.west) ++(-.5,0) node[left] {\(u\)} -- (G.west);
				\draw[ultra thick, ->] (G.east) -- ++(.5,0) node[right] {\(y\)};
			\end{tikzpicture}
			&
			T\dot{y} + y = K\dot{u} & \frac{j\omega K}{1 + j\omega T}
			\\

			\bottomrule[2pt]
		\end{longtable}
	\end{table}
\end{landscape}

\newpage

\begin{longtable}{|c|c|l|}
	\specialrule{2pt}{0pt}{0pt}
	{\bf Typ} & {\it Symbol} & {\it Gleichung, DGL}\\
	 & & {\it Sprungantwort}\\
	 & & {\it Frequenzgang, Betrag und Argument}\\ \cline{2-3}
	 & Strukturbild & {\it Nyquistdiagramm} -- {\it Bodediagramm}\\
	\specialrule{2pt}{0pt}{0pt}
	
	
	%P-Glied
	P
	&
	\begin{tikzpicture}
		\node[rtprop] (P) {};
		\node[anchor = south west] at (P.north west) {\(K\)};
		\draw[ultra thick, ->] (P.west) ++(-.5,0) node[left] {\(u\)} -- (P.west);
		\draw[ultra thick, ->] (P.east) -- ++(.5,0) node[right] {\(y\)};
	\end{tikzpicture}
	&
	% {\[\!\begin{aligned*}
	% 	y &= Ku \\
	% 	y_\varepsilon &= K\varepsilon \\
	% 	G(j\omega) &= K & |G| &= K & \arg G = 0 \\
	% \end{aligned}\]}
	\\
	\cline{2-3}
	& \parbox[c][2cm]{3cm}{\begin{tikzpicture}
  % Box
  \begin{scope}[very thick]
    \draw[->] (0,0) -- +(0.5,0) node[near start, above] {u};
    \draw (0.5,-0.6) rectangle +(1.5,1.2);
    \draw[->] (2,0) -- +(0.5,0) node[near end, above] {y};
  \end{scope}

  % Zugemüse
  \node at (0.7,0.9) {K};
  \draw[very thick] (0.5, 0.35) -- +(1.5,0);
\end{tikzpicture}}
	& 
	\parbox[c]{3cm}{\usepgflibrary{shapes.misc}
\begin{tikzpicture}
\draw[->, thick] (-0.5,0) -- (2,0) node[below] {Re};
\draw[->, thick] (0,-0.3) -- (0,1) node[left] {Im};
\node[rounded rectangle, draw=blue, thick]  at (1,0) {};
\node at (1,0.4) {K};
\node at (0.8,-0.8) {\small{frequenzunabhängig}};
\end{tikzpicture}} \quad
	\parbox[c]{6cm}{\begin{tikzpicture}[xscale=1.5, yscale=0.8]

%% Amplitude
\begin{scope}
	%% Koordinatensystem
	\draw[thick, ->] (-0.1,0) node[left] {$0$} -- (3.5,0) node[right] {$\omega T_0$};
	\draw[thick, ->] (0,-1) -- (0,0.5) node[right] {$|G|_{dB}$};
	\draw[dashed] (0,-0.6) node[left] {$-20$} -- (3.5,-0.6);
	\draw[dashed] (1,-1) -- +(0,1.2);
	\draw[dashed] (2,-1) -- +(0,1.2);
	\draw[dashed] (3,-1) -- +(0,1.2);
	%%%%%%%%%%%%%%%%

	%% Amplitudengang
	\draw[blue, very thick] (0,-0.3) -- +(3.3,0);
	\draw[<-, thick] (1.3,-0.2) -- +(0.5,0.6) node[right] {$K<1$};
\end{scope}


%% Phase
\begin{scope}[shift={(0,-1.8)}]
	%% Koordinatensystem
	\draw[thick, ->] (-0.1,0) node[left] {$0^\circ$} -- (3.5,0) node[right] {$\omega T_0$};
	\draw[thick, ->] (0,-1) -- (0,0.5) node[right] {$arg(G)=0$};
	\draw[dashed] (0,-0.6) node[left] {$-90^\circ$} -- (3.5,-0.6);
	\draw[dashed] (1,-1) node[below] {$0.1$} -- +(0,1.2);
	\draw[dashed] (2,-1) node[below] {$1$} -- +(0,1.2);
	\draw[dashed] (3,-1) node[below] {$10$} -- +(0,1.2);
	%%%%%%%%%%%%%%%%

	%% Amplitudengang
	\draw[blue, very thick] (0,0) -- +(3.3,0);
\end{scope}

%\draw (current bounding box.south west) rectangle (current bounding box.north east);

\end{tikzpicture}}			 
	\\
	\specialrule{2pt}{0pt}{0pt}
	
	
	%I-Glied
	I & \parbox[c][2cm]{3cm}{\input{./tikz/IGlied}}
	&
	\begin{tabular}{lll}
		$\dot{y} = Ku$					
		& \multicolumn{2}{l}{$y = K \int\limits_{0}^{t}u(\tau)\;d\tau \qquad y(0) = 0 \qquad [K] = sec^{-1}$}										\\
		$u=1(t)$						& $y=K t$								& \\
		$G(j \omega)=\frac{K}{j\omega}$ & $\left| G \right| = \frac{K}{\omega}$ & $arg(G)=-\frac{\pi}{2}$ \\
	\end{tabular}
	\\ \cline{2-3}
	& \parbox[c][2cm]{3cm}{\input{./tikz/IGliedStruktur}}
	&
	\parbox[c]{3cm}{\begin{tikzpicture}
\draw[->, thick] (-0.2,0) node[below] {0} -- (2,0) node[below] {Re};
\draw[->, thick] (0,-1) -- (0,0.5) node[left] {Im};

\draw[->, blue, thick] (0,-1.4) node[above right, black] {$\omega \rightarrow 0$} 
	-- (0,-0.2) node[right, black] {$\omega \rightarrow \infty$};
%\draw (current bounding box.south west) rectangle (current bounding box.north east);
\end{tikzpicture}}
	\parbox[c]{6cm}{\input{./tikz/IGliedBode}} 
	\\
	\specialrule{2pt}{0pt}{0pt}
	
	
	%D-Glied
	D & \parbox[c][2cm]{3cm}{\input{./tikz/DGlied}}
	&
	\begin{tabular}{lll}
		$y = K\dot{u}$					
		&	$[K] =sec$					& \\
		$u=1(t)$						& $y=K \delta (t)$						& \\
		$G(j \omega)=K j\omega$			& $\left| G \right| = K\omega$			& $arg(G)=\frac{\pi}{2}$
	\end{tabular}
	\\ \cline{2-3}
	& \parbox[c][2cm]{3cm}{\input{./tikz/DGliedStruktur}}			
	&
	\parbox[c]{3cm}{\begin{tikzpicture}
\draw[->, thick] (-0.2,0) node[below] {0} -- (2,0) node[below] {Re};
\draw[->, thick] (0,-0.2) -- (0,1) node[left] {Im};

\draw[->, blue, thick] (0,0) node[above right, black] {$\omega \rightarrow 0$} 
	-- (0,0.8) node[right, black] {$\omega \rightarrow \infty$};
\draw[] (0.9,-0.6) node {\footnotesize{Nicht realisierbar}};
%\draw (current bounding box.south west) rectangle (current bounding box.north east);
\end{tikzpicture}}
	\parbox[c]{6cm}{\begin{tikzpicture}[xscale=1.5, yscale=0.8]
	%% Amplitude
\begin{scope}
	%% Koordinatensystem
	\draw[thick, ->] (0,-1.2) -- (0,0.5) node[right] {$|G|_{dB}$};
	\draw[thick, ->] (-0.1,0.1) node[left] {$0$} -- +(3.6,0) node[right] {$\omega T_0$};
	\draw[dashed,gray] (0,-0.9) node[left, black] {$-40$} -- +(3.5,0);
	\draw[dashed,gray] (0,-0.4) node[left, black] {$-20$} -- +(3.5,0);
	\draw[dashed,gray] (1,-1.2) -- +(0,1.5);
	\draw[dashed,gray] (2,-1.2) -- +(0,1.5);
	\draw[dashed,gray] (3,-1.2) -- +(0,1.5);
	%%%%%%%%%%%%%%%%

	%% Amplitudengang
	\draw[blue, very thick] (0,-1.1) -- +(3.3,1.55) node[midway, below, rotate=18, black] {$20\frac{dB}{DK}$};
	\draw(2.55,0.1) node[draw,circle,inner sep=2pt,fill, blue] {};
	\node at (2.55, 0.1) [anchor = south east, blue] {$\omega = \frac{1}{K}$};
\end{scope}


%% Phase
\begin{scope}[shift={(0,-2.2)}]
	%% Koordinatensystem
	\draw[thick, ->] (-0.1,-0.3) node[left] {$0^\circ$} -- +(3.5,0) node[right] {$\omega T_0$};
	\draw[thick, ->] (0,-1) -- (0,0.5) node[right] {$argG$};
	\draw[dashed,gray] (0,0.2) node[left, black] {$90^\circ$} -- +(3.5,0);
	\draw[dashed,gray] (0,-0.8) node[left, black] {$-90^\circ$} -- +(3.5,0);
	\draw[dashed, gray] (1,-1) node[below, black] {$0.1$} -- +(0,1.2);
	\draw[dashed,gray] (2,-1) node[below, black] {$1$} -- +(0,1.2);
	\draw[dashed,gray] (3,-1) node[below, black] {$10$} -- +(0,1.2);
	%%%%%%%%%%%%%%%%

	%% Amplitudengang
	\draw[blue, very thick] (0,0.2) -- +(3.3,0);
\end{scope}

%\draw (current bounding box.south west) rectangle (current bounding box.north east);
\end{tikzpicture}} 
	\\
	\specialrule{2pt}{0pt}{0pt}
	
	\newpage
	
	\specialrule{2pt}{0pt}{0pt}
	
	%PT1_Glied
	$PT_1$ & \parbox[c][2cm]{3cm}{\input{./tikz/PT1Glied}}
	&
	\begin{tabular}{lll}
		$T\dot{y}+y=Ku$							& $y(0)=0$									& \\
		$u=1(t)$								& $y=K \left[ 1-e^{- \frac{t}{T}}\right]$	& \\
		$G(j \omega)= \frac{K}{1+j\omega T}$	& $\left| G \right| = \frac{K}{\sqrt{1+(\omega T)^2}}$ &
		$arg(G)=-\arctan(\omega T)$
	\end{tabular}
	\\ \cline{2-3}
	& \parbox[c][2cm]{3cm}{\begin{tikzpicture}
  \begin{scope}[very thick]
    \draw[->] (0,0) -- +(0.5,0) node[near start, above] {u};
    \draw (0.5,-0.6) rectangle +(1.5,1.2);
    \draw[->] (2,0) -- +(0.5,0) node[near end, above] {y};
  \end{scope}

 % Zugemüse
 \node at (0.7,0.9) {K};
 \node at (1.8,0.9) {T};
 \draw[very thick] (0.5,-0.6)  .. controls (0.7,0.2) and (1.4,0.5) .. (2,0.5);

\end{tikzpicture}}	
	&
	\parbox[c]{3.7cm}{\usepgflibrary{shapes.misc}
\begin{tikzpicture}
\draw[->, thick] (-0.9,0) -- (2,0) node[below] {Re};
\draw[->, thick] (0,-1) -- (0,0.4) node[left] {Im};

\node at (1.6,0.2) {K};
\draw[blue, thick, ->] (1.6,0) arc (0:-180:0.8);

\scalebox{0.7}{
  \node at (-0.7,-0.3) {$\omega \rightarrow \infty$};
  \node at (1.5, -0.3) {$\omega = 0$};
}
\end{tikzpicture}}
	\parbox[c]{6cm}{\begin{tikzpicture}[xscale=1.5, yscale=0.8]
	%% Amplitude
\begin{scope}
	%% Koordinatensystem
	\draw[thick, ->] (0,-1.2) -- (0,0.5) node[right] {$|G|_{dB}$};
	\draw[dashed,gray] (1,-1.2) -- +(0,1.5);
	\draw[dashed,gray] (2,-1.2) -- +(0,1.5);
	\draw[dashed,gray] (3,-1.2) -- +(0,1.5);
	
	\draw[thick, ->] (-0.1,0.1) node[left] {$0$} -- +(3.6,0) node[right] {$\omega T_0$};
	\draw[dashed,gray] (0,-0.9) node[left, black] {$-40$} -- +(3.5,0);
	\draw[dashed,gray] (0,-0.4) node[left, black] {$-20$} -- +(3.5,0);

	%%%%%%%%%%%%%%%%

	%% Amplitudengang
	\draw[blue, very thick] (0,0.1) .. controls (1.3,0.1) .. (3.5,-0.8) 
	node[very near end, below, rotate=-14, color=black] {$-20 \frac{db}{DK}$};
	\draw (1.3,0.1) -- (3.5, -0.8);
	\draw(1.3,0.1) node[draw,circle,inner sep=1pt,fill] {};
	\node at (1.3, 0.1) [anchor = south west] {$\omega = \frac{1}{T}$};
\end{scope}


%% Phase
\begin{scope}[shift={(0,-2.2)}]
	%% Koordinatensystem
	\draw[thick, ->] (0,-1) -- (0,0.5) node[right] {$argG$};
	\draw[dashed, gray] (1,-1) node[below, black] {$0.1$} -- +(0,1.2);
	\draw[dashed,gray] (2,-1) node[below, black] {$1$} -- +(0,1.2);
	\draw[dashed,gray] (3,-1) node[below, black] {$10$} -- +(0,1.2);
	
	\draw[thick, ->] (-0.1,0.2) node[left] {$0^\circ$} -- +(3.5,0) node[right] {$\omega T_0$};
	\draw[dashed,gray] (0,-0.3) node[left, black] {$-90^\circ$} -- +(3.5,0);
	\draw[dashed,gray] (0,-0.8) node[left, black] {$-180^\circ$} -- +(3.5,0);

	%%%%%%%%%%%%%%%%

	%% Amplitudengang
	\draw[blue, very thick] (0,0.2) .. controls (2.0,0.2) and (0.6,-0.3) .. (2.6,-0.3) -- +(0.9,0);
	\draw (0.3,0.2) -- (2.3,-0.3);
	
\end{scope}

%\draw (current bounding box.south west) rectangle (current bounding box.north east);
\end{tikzpicture}}
	\\
	\specialrule{2pt}{0pt}{0pt}
	
	
	%PT2_Glied
	$PT_2$ &
	\begin{minipage}{3cm}
		\begin{tikzpicture}
  \begin{scope}[very thick]
    \draw[->] (0,0) -- +(0.5,0) node[near start, above] {u};
    \draw (0.5,-0.6) rectangle +(1.5,1.2);
    \draw[->] (2,0) -- +(0.5,0) node[near end, above] {y};
  \end{scope}

  \node at (0.7,0.9) {K};
  \node at (1.8,0.9) {D,$\omega_0$};

  % Inhalt
  \begin{scope}[shift={(0.8,-0.4)}]
    \draw[->, thick] (-0.2,0) -- +(1.3,0);
    \draw[->, thick] (0,-0.1) -- +(0,1);

    \draw[blue, very thick] (-0.2,0) -- ++(0.2,0)
              .. controls (0.15,1.6)  and (0.35,-0.1) .. (0.50,0.6)
              .. controls (0.60,0.8)  and (0.7,0.4) .. (0.8,0.6)
              .. controls (0.85,0.65) and (0.88,0.61) .. (0.9,0.6);
  \end{scope}

\end{tikzpicture}
	\end{minipage}
	&
	\begin{tabular}{lll}
		$T^2\ddot{y}+2\zeta T \dot{y}+y=Ku$ & $\ddot{y}+2\zeta\omega_n \dot{y}+\omega_n^2y=K\omega_n^2u$ & \\
		$y(0)=0$ & $\dot{y}(0)=0$ & $\omega_n=\frac{1}{T}$ \\
		\multicolumn{3}{l}{
			$y=K \left[1-\frac{1}{\sqrt{1-\zeta^2}}e^{-\zeta\omega_n t}\sin
			\left( \sqrt{1-\zeta^2} \omega_n t+arcos(\zeta) \right)\right]$
		} \\
		$G(j \omega)= \frac{K}{1+ 2 \zeta (j\omega) T  + (j \omega T)^2}$ & $\left| G \right| = \frac{K}{\sqrt{\left[1+(j\omega
		T)^2\right]^2+\left[2\zeta \omega T \right]^2}}$ & \\
		$\arg(G)=-\arctan  \frac{2\zeta \omega T}{1+(j\omega T)^2}$ & $0 \leq\omega T \leq 1$ & \\
		$\arg(G)=\arctan \frac{2\zeta \omega T}{1+(j \omega T)^2}-\pi$ & $1 \leq\omega T \leq \infty$ & \\
		
	\end{tabular}
	\\ \cline{2-3}
	& \parbox[c][2cm]{3cm}{\begin{tikzpicture}
  \begin{scope}[very thick]
    \draw[->] (0,0) -- +(0.5,0) node[near start, above] {u};
    \draw (0.5,-0.6) rectangle +(1.5,1.2);
    \draw[->] (2,0) -- +(0.5,0) node[near end, above] {y};
  \end{scope}

   \node at (0.7,0.9) {K};
    \node at (1.25,0.9) {$\zeta$};
    \node at (1.8,0.9) {T};

  % Inhalt
  \begin{scope}[shift={(0.8,-0.4)}]

	\draw[very thick] (-0.3,-0.2)
              .. controls (-0.1,-0.2) .. (0.0,0.5)
              .. controls (0.2,1.6)  and (0.3,-0.1) .. (0.5,0.7)
              .. controls (0.65,1.1)  and (0.75,0.5) .. (0.9,0.6)
              .. controls (1.0,1.0) and (1.1, 0.6) ..
              (1.2, 0.55)
              ;
  \end{scope}
\end{tikzpicture}}
	& \begin{minipage}{3cm}
	\includegraphics[angle = {-0.3}, width=3cm]{./images/PT2_Nyq.jpg}
	\end{minipage}
	\begin{minipage}{9cm}
	\includegraphics[angle = {0.2}, width=8cm]{./images/PT2_Bode.jpg}
	\end{minipage} \rule[-5mm]{0mm}{35mm}
	\\
	\specialrule{2pt}{0pt}{0pt}
	
	
	%Tt_Glied
	$T_t$ &
	\parbox[c][2cm]{3cm}{\input{./tikz/TtGlied}}
	&
	\begin{tabular}{lll}
		$y=\begin{cases}
			0 & 0<t<T_t \\
			u(t-T_t) & t \geq T_t
			\end{cases}$ & & \\
		$u=1(t)$ & $y=1(t-T_t)$ & \\
		$G(j \omega)= e^{-j\omega T_t}$ & $\left| G \right| = 1$ & $arg(G)=-\omega T_t$
	\end{tabular}
	\\ \cline{2-3}
	& \parbox[c][2cm]{3cm}{\begin{tikzpicture}
  \begin{scope}[very thick]
    \draw[->] (0,0) -- +(0.5,0) node[near start, above] {u};
    \draw (0.5,-0.6) rectangle +(1.5,1.2);
    \draw[->] (2,0) -- +(0.5,0) node[near end, above] {y};
  \end{scope}

% Zugemüse
\node at (0.7,0.9) {K};
\node at (1.8,0.9) {D,$\omega_0$};
\draw[very thick] (0.9,-0.6) -- ++(0,1) -- +(1.1,0);
\end{tikzpicture}}
	& 
	\parbox[c]{3cm}{\usetikzlibrary{decorations.markings}
\begin{tikzpicture}
\useasboundingbox (-1.4,-1.4) rectangle (1.9,1.4);
\draw[->, thick] (-1.1,0) -- (1.6,0) node[above] {Re};
\draw[->, thick] (0,-1) -- (0,1.1) node[left] {Im};

\draw (0.8,0.05) arc(90:450:0.05) node[above right] {\small{1}};
\draw (-0.8,0.05) arc(90:450:0.05) node[above left] {\small{-1}};

\draw[blue, thick,
	decoration={
		markings,
		mark= at position 0.9 with {\arrow{<}}
	},
	postaction={decorate}
] (0.8,0) arc (0:360:0.8);

\scalebox{0.7}{
  \node at (1.3,-1.5) {$2\pi$-periodisch};
  \node at (1.6, -0.3) {$\omega = 0$};
}
%\draw (current bounding box.south west) rectangle (current bounding box.north east);
\end{tikzpicture}}
	\parbox[c]{6cm}{\begin{tikzpicture}[xscale=1.5, yscale=0.8]
	%% Amplitude
\begin{scope}
	%% Koordinatensystem
	\draw[thick, ->] (0,-1.2) -- (0,0.5) node[right] {$|G|_{dB}$};
	\draw[dashed,gray] (1,-1.2) -- +(0,1.5);
	\draw[dashed,gray] (2,-1.2) -- +(0,1.5);
	\draw[dashed,gray] (3,-1.2) -- +(0,1.5);
	
	\draw[thick, ->] (-0.1,0.1) node[left] {$0$} -- +(3.6,0) node[right] {$\omega T_0$};
	\draw[dashed,gray] (0,-0.9) node[left, black] {$-40$} -- +(3.5,0);
	\draw[dashed,gray] (0,-0.4) node[left, black] {$-20$} -- +(3.5,0);

	%%%%%%%%%%%%%%%%

	%% Amplitudengang
	\draw[blue, very thick] (0,0.1) -- +(3.4,0);
\end{scope}


%% Phase
\begin{scope}[shift={(0,-2.2)}]
	%% Koordinatensystem
	\draw[thick, ->] (0,-1) -- (0,0.5) node[right] {$arg(G)$};
	\draw[dashed, gray] (1,-1) node[below, black] {$0.1$} -- +(0,1.2);
	\draw[dashed,gray] (2,-1) node[below, black] {$1$} -- +(0,1.2);
	\draw[dashed,gray] (3,-1) node[below, black] {$10$} -- +(0,1.2);
	
	\draw[thick, ->] (-0.1,0.2) node[left] {$0^\circ$} -- +(3.5,0) node[right] {$\omega T_0$};
	\draw[dashed,gray] (0,-0.3) node[left, black] {$-90^\circ$} -- +(3.5,0);
	\draw[dashed,gray] (0,-0.8) node[left, black] {$-180^\circ$} -- +(3.5,0);

	%%%%%%%%%%%%%%%%

	%% Amplitudengang
	\draw[blue, very thick] (0,0.2) .. controls (0.8,0.2) and (1.7,0.2) .. (2.3,-1.4);	
\end{scope}

%\draw (current bounding box.south west) rectangle (current bounding box.north east);
\end{tikzpicture}}
	\\
	\specialrule{2pt}{0pt}{0pt}

%DT1_Glied
	$DT_1$ &
	\parbox[c][2cm]{3cm}{\begin{tikzpicture}
  \begin{scope}[very thick]
    \draw[->] (0,0) -- +(0.5,0) node[near start, above] {u};
    \draw (0.5,-0.6) rectangle +(1.5,1.2);
    \draw[->] (2,0) -- +(0.5,0) node[near end, above] {y};
  \end{scope}

 % Zugem�se
 \node at (0.7,0.9) {$T_V$};
 \node at (1.8,0.9) {$T_C$};
 \draw[very thick] (0.5,0.5)  .. controls (0.7,-0.1) and (1.4,-0.4) .. (2,-0.5);

\end{tikzpicture}}
	&
	\begin{tabular}{lll}
		$G(j \omega)= G_D \cdot G_{PT_1} = j\omega T_V \dfrac{1}{1+j\omega T_C} = \dfrac{T_V}{T_C}\left(1- \dfrac{1}{1+ j\omega T_C} \right)$ \\
	\end{tabular}\\
	\specialrule{2pt}{0pt}{0pt}
		\end{longtable}



\subsection{D-Glied  / $\text{DT}_1$-Glied}
Der Differenzierer erzeugt ein Korrektursignal im voraus.
Nachteilig ist, wenn die Regelgrösse verrauscht ist, dann werden die hochfrequenten Störsignale durch die Ableitung verstärkt.
Ein LTI-System, welches ohne D-Glied darstellbar ist, gegebenenfalls durch Umformung des Blockdiagramms, heisst realisierbar.
In der Realität wird meistens kein reines D-Glied sondern ein $DT_1$-Glied verwendet:

\begin{tabular}{|l||lll| l}
  \cline{1-4}
	\parbox[c][2cm]{3cm}{\begin{tikzpicture}
  \begin{scope}[very thick]
    \draw[->] (0,0) -- +(0.5,0) node[near start, above] {u};
    \draw (0.5,-0.6) rectangle +(1.5,1.2);
    \draw[->] (2,0) -- +(0.5,0) node[near end, above] {y};
    \draw(0.9, -0.2) rectangle +(1.1,0.8);
  \end{scope}

  \node at (0.7,0.9) {$K_D$};

\end{tikzpicture}} &
	\parbox[c][2cm]{4.5cm}{\begin{tikzpicture}
  %D-Glied
  \begin{scope}[very thick]
    \draw[->] (0,0) -- +(0.5,0) node[near start, above] {};
    \draw (0.5,-0.6) rectangle +(1.5,1.2);
    \draw[->] (2,0) -- +(0.5,0) node[near end, above] {};
    \draw(0.9, -0.2) rectangle +(1.1,0.8);
  \end{scope}

  \node at (0.7,0.9) {$T_V$};

  %PT1-Glied
  \begin{scope}[very thick]
    \draw (2.5,-0.6) rectangle +(1.5,1.2);
    \draw[->] (4,0) -- +(0.5,0) node[near end, above]{};
  \end{scope}

   % Zugem�se
   \node at (2.7,0.9) {1};
   \node at (3.8,0.9) {$T_C$};
   \draw[very thick] (2.5,-0.6)  .. controls (2.7,0.2) and (3.4,0.5) .. (4,0.5);

\end{tikzpicture}} &
	$\Rightarrow$ &
	\parbox[c][2cm]{3cm}{\begin{tikzpicture}
  \begin{scope}[very thick]
    \draw[->] (0,0) -- +(0.5,0) node[near start, above] {u};
    \draw (0.5,-0.6) rectangle +(1.5,1.2);
    \draw[->] (2,0) -- +(0.5,0) node[near end, above] {y};
  \end{scope}

 % Zugem�se
 \node at (0.7,0.9) {$T_V$};
 \node at (1.8,0.9) {$T_C$};
 \draw[very thick] (0.5,0.5)  .. controls (0.7,-0.1) and (1.4,-0.4) .. (2,-0.5);

\end{tikzpicture}} &
	\fbox{$G_{DT_1}(j\omega) = \frac{j\omega T_V}{1+ j\omega T_C} = \frac{T_V}{T_C}\left(1- \frac{1}{1+j\omega T_C} \right) $}
	\\
	$D$-Glied &
	$D$-Glied \qquad $PT_1$-Glied & &
	$DT_1$-Glied \\
  \cline{1-4}
\end{tabular}

\subsection{PT$_2$-Glied}
\subsubsection{Parameter der Sprungantwort}
\renewcommand{\arraystretch}{1.8}
\begin{tabular}{|m{7cm}|m{1cm}m{0.5cm}m{8cm}}
  \cline{1-1}
  $y_{\infty} =A \cdot K$ & &
  $y_{\infty}$: & Endwert\\
  \cline{1-1}  
	$T_\omega = 2T_m=\dfrac{2\pi}{\omega_n \sqrt{1-\zeta^2}}=\dfrac{2\pi}{\omega}$ & &
	$T_{\omega}$: & Schwingungsdauer \\
  \cline{1-1}
	$\varepsilon = \dfrac{\Delta y}{y_{\infty}}$ & &
	\parbox{0.5cm}{
		$\varepsilon$:\\
		$\Delta y$:
	} & 
	\parbox{8cm}{
		Verhältnis von Überschwinger nach $T_e$ zum Endwert\\
		Toleranzbereich der Amplitude nach $T_e$
	}\\
  \cline{1-1}  
	$T_e = \dfrac{\ln\left(\varepsilon\sqrt{1-\zeta^2}\right)}{-\omega_n\cdot\zeta} = 
	\dfrac{1}{\sigma}\ln\left(\dfrac{\varepsilon\omega}{\omega_n}\right)$ & &
	$T_e$: & Einschwingzeit \\
  \cline{1-1}  
	$T_m = \dfrac{\pi}{\omega_n\sqrt{1-\zeta^2}}=\dfrac{\pi}{\omega}$ & &
	$T_m$: & Überschwingungsdauer\\
  \cline{1-1}  
	$y_m = y_{\infty} \cdot e^{\frac{-\pi\cdot\zeta}{\sqrt{1-\zeta^2}}}$ & &
	$y_m$: & Überschwingweite\\
   \cline{1-1}  
	$\omega = \dfrac{1}{T}\sqrt{1-\zeta^2}= \omega_n\sqrt{1-\zeta^2}=\dfrac{2\pi}{T_\omega}=2\pi f$ & &
	$\omega$: & Kreisfrequenz \\
  \cline{1-1}  
	$\omega_n = \dfrac{1}{T}$ & &
	$\omega_n$: & Kennkreisfrequenz \\
  \cline{1-1}  
	$T_a = \frac{\pi - \arccos{(\zeta)}}{\omega_n\cdot\sqrt{1-\zeta^2}}$ & &
	$T_a$: & Anschwingzeit/Anregelzeit \\
  \cline{1-1}  
	$\sigma = -\dfrac{\zeta}{T} = -\zeta\omega_n$ & &
	$\zeta$: & Dämpfungskonstante \\
  \cline{1-1}
	$\delta = \ln{\Big(\frac{y_i}{y_{i+1}}\Big)} = \frac{2\pi \zeta}{\sqrt{1-\zeta^2}}$ & &
	$\delta$: & Logarithmisches Dekrement\\
  \cline{1-1}  
\end{tabular}
\renewcommand{\arraystretch}{1}
	
\subsubsection{Sprungantwort}
\includegraphics[width = 9cm]{./images/pt2StepResp}

\subsubsection{Dämpfung}
Optimal bei $\zeta=\frac{1}{\sqrt{2}}$ ($\Psi=45$).
Dabei erreicht die Regelgrösse $y$ nach $4.3\%$ Überschwingen rasch den	Endwert.

\subsubsection{Berechnung $\zeta$}
\textbf{Aus DGL} $\ddot{y}+a_1\dot{y}+a_0 y=\ldots$ folgt $a_1=2\zeta\omega_n$, 
$a_0=\omega_n^2$
$\Rightarrow \zeta=\frac{a_1}{2\sqrt{a_0}}$ \\
\textbf{Mittels Überschwingweite} kann $\zeta$ ebenfalls berechnet werden\\
\begin{tabular}{p{2.5cm}p{2.5cm}p{4cm}}
$\zeta = \frac{1}{\sqrt{1+(\frac{\pi}{c})^2}}$ & $c =ln(\frac{y_m}{y_{\infty}})$ & $y_m$: Überschwingweite
\end{tabular}

Weitere Formeln in der LTI-Grundglieder Tabelle
 %% TODO: fix this table
\twocolumn
\section{Stabilitätsproblem \formelbuch{111}}
Zur Instabilität führt das Zusammenwirken von Verstärkung und Signalverzögerung. Dabei ist die Reihenfolge der Glieder beliebig. Entscheidend sind gesamte Verstärkung ($=$ Kreisverstärkung) und gesamte Verzögerung ($=$ Kreisverzögerung) im offenen Regelkreis.

\begin{figure}[h!]
	\centering
	\begin{tikzpicture}
		\matrix[column sep = 8mm, ampersand replacement=\&]{
			\node[rtsum] (S) {}; \node[left = 6mm] (u) {\(u\)}; \&
			\node[rtprop] (P) {}; \&
			\node[rtdelay] (T) {}; \&
			\node[rtsplit] (Y) {}; \node[right = 6mm] (y) {\(y\)}; \\
			};

			\node[anchor = south west] at (P.north west) {\(K_P\)};
			\node[anchor = south east] at (T.north east) {\(T_t\)};

			\draw[very thick, ->]
			(u) edge node[below, pos = .5] {\(+\)} (S)
			(S) edge (P)
			(P) edge (T)
			(T) edge (Y)
			(Y) edge (y);

			\draw[very thick, ->]
			(Y) |- ($(S) - (0,.8)$) -- node[left, pos = .5] {\(-\)} (S);
	\end{tikzpicture}
\end{figure}
P-Glied mit Totzeit gegengekoppelt hat \(G_0(j\omega) = K_P e^{-j\omega T_t}\).
\begin{itemize}
	\item \(K_P < 1\) stabil
	\item \(K_P = 1\) grenzstabil
	\item \(K_P > 1\) instabil mit Phasenschnittfrequenz \(\omega_\pi = 2\pi/T_t\)
\end{itemize}

\begin{figure}[h!]
	\centering
	\begin{tikzpicture}
		\matrix[column sep = 8mm, ampersand replacement=\&]{
			\node[rtsum] (S) {}; \node[left = 6mm] (u) {\(u\)}; \&
			\node[rtint] (I) {}; \&
			\node[rtdelay] (T) {}; \&
			\node[rtsplit] (Y) {}; \node[right = 6mm] (y) {\(y\)}; \\
		};

		\node[anchor = south west] at (I.north west) {\(K_I\)};
		\node[anchor = south east] at (T.north east) {\(T_t\)};

		\draw[very thick, ->]
			(u) edge node[below, pos = .5] {\(+\)} (S)
			(S) edge (I)
			(I) edge (T)
			(T) edge (Y)
			(Y) edge (y);

		\draw[very thick, ->]
			(Y) |- ($(S) - (0,.8)$) -- node[left, pos = .5] {\(-\)} (S);
	\end{tikzpicture}
\end{figure}
I-Glied mit Totzeit gegengekoppelt \formelbuch{109} hat \(G_0(j\omega) = \frac{K_I}{j\omega}e^{-j\omega T_t}\).
\begin{itemize}
	\item Grenzkurve \(K_I T_t = \pi/2 \)
	\item Phasenschnittfrequenz \(\omega_\pi = K_I\)
\end{itemize}

\begin{figure}[h!]
	\centering
	\begin{tikzpicture}
		\matrix[column sep = 8mm, ampersand replacement=\&]{
			\node[rtsum] (S) {}; \node[left = 6mm] (u) {\(u\)}; \&
			\node[rtint] (I1) {}; \&
			\node[rtint] (I2) {}; \&
			\node[rtsplit] (Y) {}; \node[right = 6mm] (y) {\(y\)}; \\
		};

		\node[anchor = south west] at (I1.north west) {\(K_1\)};
		\node[anchor = south west] at (I2.north west) {\(K_2\)};

		\draw[very thick, ->]
			(u) edge node[below, pos = .5] {\(+\)} (S)
			(S) edge (I1)
			(I1) edge (I2)
			(I2) edge (Y)
			(Y) edge (y);

		\draw[very thick, ->]
			(Y) |- ($(S) - (0,.8)$) -- node[left, pos = .5] {\(-\)} (S);
	\end{tikzpicture}
\end{figure}
Zwei I-Glieder gegengekoppelt \formelbuch{111}.
\begin{itemize}
	\item Phasenschnittfrequenz \(\omega_\pi = \sqrt{K_1 K_2}\)
\end{itemize}


\subsection{Spezielle Nyquistkriterium \formelbuch{125\&128}}
\begin{figure}[h!]
	\centering
	\includegraphics[width = .9\linewidth]{./images/Nyquistkurve}
	%% TODO: finish
	\iffalse
	\begin{tikzpicture}
		\pgfmathsetmacro\radius{2.25}

		% axis and unit circle
		\draw[thick, ->] (-1.25*\radius, 0) to (2.25*\radius, 0) node[right] {Re};
		\draw[thick, ->] (0, -1.25*\radius) to (0, 1.25*\radius) node[above] {Im};
		\draw[gray, thick, densely dotted] (0,0) circle (\radius);

		% scaled curves (increase k)
		\foreach \scale / \color / \domain in {%
			1.3/lightgray/1.2:{1.5*pi},
			1.6/lightgray/1.4:{1.4*pi},
			1.86/red!50!gray/1.6:{1.3*pi}%
		}{
			\draw[draw = \color, thick, domain = \domain, samples = 40] plot
				({\scale * \radius * (1/4 + \x / 4 * cos(\x * 180/pi))},
				 {\scale * \radius * \x / 4 * sin(\x * 180/pi)});
		}

		% curve
		\draw[black, ultra thick, domain = 1:{2*pi}, samples = 40] plot
			({\radius * (1/4 + \x / 4 * cos(\x * 180/pi))}, {\radius * \x / 4 * sin(\x * 180/pi)});

	\end{tikzpicture}
	\fi
\end{figure}
Der geschlossene Regelkreis ist genau dann stabil, wenn beim Durchlauf der Ortskurve des offenen Regelkreises \(G_0\) in Richtung zunehmender Frequenz der kritische Punkt \(-1\) ``zur Linken'' liegt, daher nicht umschlungen wird.

Dies ist eine vereinfachte Form des Nyquist-Kriteriums und setzt einen stabilen offenen Kreis voraus (Prozess mit Ausgleich), der auch noch durch ein I-Glied ergänzt sein darf (Prozess ohne Ausgleich).

\subsection{Phasenreserve und Verstärkungsreserve \formelbuch{128-129}}

\begin{figure}[h!]
	\centering
	% \includegraphics[width = .9\linewidth]{./images/phasenreserve.png}
	\begin{tikzpicture}
		\pgfmathsetmacro\radius{2.25}

		% axis and unit circle
		\draw[thick, ->] (-1.5*\radius, 0) to (1.5*\radius, 0) node[right] {Re};
		\draw[thick, ->] (0, -1.25*\radius) to (0, 1.25*\radius) node[above] {Im};
		\draw[gray, thick, densely dotted] (0,0) circle (\radius);

		% scaled curves (increase k)
		\foreach \scale / \color / \domain in {%
			1.3/lightgray/0:2.1,
			1.6/lightgray/0:1.9,
			2.03/red!50!gray/0:1.7%
		}{
			\draw[draw = \color, thick, domain = \domain, samples = 40] plot
				({\scale * \radius * -1 * \x / 3 * sin(\x * 180 / pi - 20)}, 
				 {\scale * \radius * \x / 2 * cos(\x * 180 / pi)});
		}

		% instability at -1
		\node[
			circle,
			draw = red!50!gray, thick, solid,
			fill = white,
			minimum size = 4pt,
			inner sep = 0pt,
			label = {[text = red!50!gray]120:\(-1\)},
		] at (-1*\radius,0) {};

		\node[
			red!65!black,
			fill = white,
			text width = 3cm,
		] at (-\radius, \radius) {%
			Ortskurve an der Stabilit\"atsgrenze%
			% \(K_{R\pi}G_0(j\omega_\pi) = -1\)
		};

		% curve
		\draw[black, ultra thick, domain = 0:2.5, samples = 40] plot
			({\radius * -1 * \x / 3 * sin(\x * 180 / pi - 20)},
			 {\radius * \x / 2 * cos(\x * 180 / pi)});

		\node[below right] at (230:1.3*\radius) {\(G_0(j\omega)\)};
		\node[above right] at (0,0) {\(\omega \to \infty\)};

		% phase reserve
		\draw[thick, dashed, gray] (0,0) to (225:1*\radius)
			node[
				circle,
				draw = black, thick, solid,
				fill = white,
				minimum size = 4pt,
				inner sep = 0pt,
				label = {[text = black]0:\(\omega_D\)},
			] {}
			to (225:1.5*\radius);

		\draw[thick, <->] (225:1.3*\radius) arc (225:180:1.3*\radius)
			node[left, pos = .4] {\(\Phi_\text{res}\)};

		% amplification reserve
		\node[
			circle,
			draw = black, thick, solid,
			fill = white,
			minimum size = 4pt,
			inner sep = 0pt,
			label = {[text = black, label distance = -2pt]-45:\(\omega_\pi\)},
		] at (-.49*\radius,0) {};

		% few more infos
		\matrix[
			anchor = west,
			nodes = {
				fill = white,
				text = black,
			},
		] at (.2*\radius, 0) {
			\node {\(|G_0(j\omega_D)| = 1\)}; \\
			\node {\(\arg G_0(j\omega_\pi) = - \pi\)}; \\[20mm]
			\node {\(G_0(j\omega_\pi) = -1/K_\text{res}\)}; \\
			\node {\(\arg G_0(j\omega_D) = -\pi + \Phi_\text{res}\)}; \\
		};

	\end{tikzpicture}
\end{figure}

% \subsubsection{Vorgehen beim Einstellen von P-Regler}
% \textbf{Gegeben \(\Phi_\text{res}\) finde \(K_\text{res}\)}
% \begin{enumerate}
% 	\item Mit \(\arg G_0(\omega_D) = -\pi + \Phi_\text{res}\) Durchtrittsfrequenz \(\omega_D\) bestimmen.
% 	\item Dann \(K_\text{res} \cdot |G_0(j\omega_D)| = 1\) nach \(K_\text{res}\) aufl\"osen.
% 	\item Das P-Glied muss mindestens \(K_R \leq 1 / K_\text{res}\) haben.
% \end{enumerate}
% 
% \textbf{Gegeben \(K_{\text{res}}\) finde  \(\Phi_\text{res}\)}
% \begin{enumerate}
% 	\item Mit \(1 = K_\text{res} \cdot |G_0(j\omega_\pi)|\) Phasenschnittfrequenz \(\omega_\pi\) bestimmen.
% 	\item 
% \end{enumerate}

\subsection{Stabilit\"at in Bodediagramme \formelbuch{140}}

\subsection{Alternative stabilit\"atskriterien \formelbuch{141}}

Eine \emph{notwendige} Stabilit\"atsbedingung f\"ur das charakteristische Polynom
\[
	a_n \lambda^n + \cdots + a_2 \lambda^2 + a_1 \lambda + a_0
\]
besteht darin, dass alle Koeffizienten \(a_n, \ldots, a_0\) dasselbe Vorzeichen haben.
Bei Systemen 1. und 2. Ordnung ist die Vorzeichenregel \emph{hinreichend} f\"ur die Stabilit\"at.

\subsection{Stabilitätssatz für die Sprungantwort}
Ein LTI-System ist genau dann stabil, wenn die Sprungantwort auf einen konstantem Wert zustrebt.

\section{PID-Regler \formelbuch{142}}

\subsection{P-Regler -- Stationärer Zustand \formelbuch{147}}
Beim einfachsten linearen Regler, dem P-Typ, besteht ein proportionaler Zusammenhang zwischen Fehler $e$ und Stellgrösse $u$.
Der P-Regler reagiert schnell, kann aber den Sprungfehler nicht vollständig eliminieren.
Er hat einen stationären Fehler.
Eine zu hohe Verstärkung $K_R$ führt zu Rauschen.


\subsection{I-Regler \formelbuch{149}}
Der reine I-Regler ist allgemein ungünstig, weil er relativ langsam arbeitet und die Stabilität schwächt. Ist aber die Regelstrecke nur erster Ordnung, erzielt man gute Ergebnisse mit dem I-Regler.
Der I-Regler neigt zum Schwingen.
Bei sprungförmigen Signalen, d.h. für Festwertregelungen hat der I-Regler keinen Fehler!


\subsection{PI-Regler \formelbuch{150}}
\begin{figure}[h!]
	\includegraphics[width=\linewidth]{./images/PI_Regler.jpg}
\end{figure}
\begin{description}
	\item[Sprungantwort]
		\[
			u(t) = K_R\left( 1 + \frac{t}{T_N}\right)
		\]
	\item[\"Ubertragungsfunktion]
		\[
			G_\text{PI}(j\omega) = K_R \frac{1+j\omega T_N}{j\omega T_N}
		\]
\end{description}


\subsection{PD-Regler}
\begin{figure}[h!]
	\includegraphics[width=\linewidth]{./images/PD_Regler.png}
\end{figure}
Der PD-Regler entspricht dem inversen PT\(_1\)-Glied. Meistens wird jedoch der PDT\(_1\) Regler verwendet.
\begin{description}
	\item[Rampenantwort]
		\[
			u(t) = K_R at + K_R T_V a \text{ wenn } e(t) = a\cdot t
		\]
	\item[Sprungantwort]
		\[
			u(t) = K_R \left( 1 + \frac{T_V}{T_C} \cdot e^{-t/T_C} \right)
			\text{ wenn } e(t) = 1(t)
		\]
	\item[\"Ubertragungsfunktion] \(T_V > T_C\)
		\[
			G_{\text{PDT}_1}(j\omega) = K_R \frac{1+j\omega(T_V+T_C)}{1+j\omega T_C}
		\]
\end{description}
 

\subsection{PIDT\(_1\)-Regler (Reeller PID) \formelbuch{153}}
Praktisch sind \(T_N \leq T_V\), \(T_C < T_V\) und \(T_C = T_V/(4\ldots \text{ bis }\ldots 50)\)

\subsubsection{Additive Form}
\begin{figure}[h!]
	\includegraphics[width = \linewidth]{./images/PID_Regler_add}
\end{figure}
\begin{description}
	\item[\"Ubertragungsfunktion]
		\[
			G_{\text{PIDT}_1}^{+} (j\omega) = K_R \left(
				1 + \frac{1}{j\omega T_N} + \frac{j\omega T_V}{1 + j\omega T_C}
			\right)
		\]
	\item[Sprungantwort]
		\[
			u(t) = K_R \left( 1 + \frac{t}{T_N} + \frac{T_V}{T_C}e^{-t/T_C} \right)
		\]
\end{description}

\subsubsection{Multiplikative Form (Serienschaltung)}
\begin{figure}[h!]
	\includegraphics[width = \linewidth]{./images/PID_Regler_mul}
\end{figure}
\begin{description}
	\item[\"Ubertragungsfunktion]
		\[
			G_{\text{PIDT}_1}^{\times} (j\omega) = 
			K_R \cdot \frac{1 + j\omega T_N}{j\omega T_N}
			\cdot \frac{1 + j\omega (T_V + T_C)}{1 + j\omega T_C}
		\]
	\item[Sprungantwort]
		\[
			u(t) = K_R \left[ 
			1 + \frac{T_V}{T_N} +\frac{t}{T_N} + \left(
			\frac{T_V}{T_C}-\frac{T_V}{T_N}
			\right) e^{-t/T_C}
			\right]
		\]
\end{description}

\subsection{Empirische Einstellregeln \formelbuch{162}}
\begin{figure}[h!]
	\includegraphics[width=\linewidth]{./images/Empirisch_Regeln.jpg}
\end{figure}
UTF des angenäherten Modells:
\[
	G_0(j\omega) = \frac{K_s}{1+j\omega T_g} e^{-j\omega T_u}
\]
Empirische Einstellregeln ergeben in der Praxis nicht immer das bestmögliche Zeitverhalten, sondern sie liefern eine erste Einstellung, welche experimentell noch verbessert werden kann.
  
Um Ausschläge im Stellsignal, welche durch die typische Reaktion eines DT\(_1\) auf einen Sprung verursacht werden, zu verhindern, darf man die Führungsgrösse \(r\) nicht über den Differenzierer leiten.

%% TODO: fix this table
\begin{table*}
	\renewcommand\arraystretch{1.2}
	\centering
	\begin{tabular}{ ll rrrrrr }
		\toprule
		\multicolumn{6}{l}{
			\textbf{Chien-Hrones-Reswick}
		}
		&
		\multicolumn{2}{l}{
			\textbf{Ziegler-Nichols}
		}
		\\ 

		\cmidrule(lr){1-6} \cmidrule(lr){7-8}
		\multicolumn{6}{l}{
			\(\displaystyle
				q = \frac{T_g}{T_uK_S}, \quad
				\mu = \frac{T_g}{T_u} \quad
				\text{wenn } \mu
				\begin{cases}
					> 10 & \text{Strecke gut regelbar} \\
					< 3  & \text{Strecke schlecht regelbar}
				\end{cases}
			\)
		}
		&
		\(\displaystyle q = \frac{T_g}{T_uK_s} \)
		&
		\(\displaystyle K_{R\pi}, \qquad T_\pi = \frac{2\pi}{\omega_\pi}\)
		\\

		\cmidrule(lr){1-6} \cmidrule(lr){7-8}

		\textbf{Regler}
		&
		% \textbf{Parameter}
		&
		\multicolumn{2}{p{3.5cm}}{
			\textbf{Führungsverhalten} % \newline $y_m$: Überschwingen
		}
		&
		\multicolumn{2}{l}{
			\textbf{Störverhalten}
		}
		&
		\multicolumn{1}{l}{
			\textbf{Sprungantwort}
		}
		&
		\multicolumn{1}{l}{
			\textbf{Stabilitätsgrenze}
		}
		\\

		\multicolumn{2}{l}{und Parameter}
		& kein $y_m$ & $y_m / y_\infty = 20 \%$ & kein $a$ & $a / b = 20 \%$ & & \\

		\cmidrule(lr){1-2}
		\cmidrule(lr){3-4}
		\cmidrule(lr){5-6}
		\cmidrule(lr){7-7}
		\cmidrule(lr){8-8}
		P & $K_R$ & $0.3q$ & $0.7q$ & $0.3q$ & $0.7q$ & $q$ & $0.5K_{R\pi}$ \\

		% \cmidrule(lr){1-6} \cmidrule(lr){7-8}
		PI & $K_R$ & $0.35q$   & $0.6q$ & $0.6q$ & $0.7q$    & $0.9q$    & $0.45K_{R\pi}$ \\
       & $T_N$ & $1.17T_g$ & $1T_g$ & $4T_u$ & $2.33T_u$ & $3.33T_u$ & $0.85T_{\pi}$  \\

		% \cmidrule(lr){1-6} \cmidrule(lr){7-8}
		PID & $K_R$ & $0.6q$   & $0.95q$   & $0.95q$   & $1.2q$    & $1.2q$   & $0.60K_{R\pi}$ \\
        & $T_N$ & $1T_g$   & $1.36T_g$ & $2.38T_u$ & $2T_u$    & $2T_u$   & $0.50T_\pi$    \\
        & $T_V$ & $0.5T_u$ & $0.47T_u$ & $0.42T_u$ & $0.42T_u$ & $0.5T_u$ & $0.125T_\pi$   \\
		\bottomrule
	\end{tabular}
	\caption{Empirische Reglereinstellung}
\end{table*}

\subsection{Wind-Up \formelbuch{170}}
\begin{description}
	\item[Definition] Der Fehler \(e\) am Integratoreingang bleibt konstant, sodass dessen Ausgangssignal ständig zunimmt.
	\item[Folge] Einerseits ein konstanter Fehler und andererseits eine verzögert reagierende und damit stark überschwingende Regelgrösse \(y\).

	\item[Ursachen]
		\begin{itemize}
			\item I-Anteil
			\item Sättigung am Regler-Ausgang
			\item \(e(t)\) über ``längere Zeit'' $\neq 0$
		\end{itemize}

	\item[Anti-Wind-Up] Integration beschränken.
\end{description}

\onecolumn

%% TODO: fix this table
\begin{table}
	\centering
	\renewcommand\arraystretch{1.2}
	\begin{tabular}{|l|p{1.8cm}|l|l|l|l||l|l|}
			\hline
			\multicolumn{6}{|c||}{
				\textbf{Reglereinstellung nach Chien-Hrones-Reswick}
			} &
			\multicolumn{2}{|c|}{
				\textbf{Reglereinstellung nach Ziegler-Nichols}
			}
		\\ \hline
		\multicolumn{6}{|c||}{
			$
			q = \frac{T_g}{T_uK_S} \qquad \mu = \frac{T_g}{T_u}
			\qquad \text{wenn} \quad \mu
			\begin{cases}
				> 10 \rightarrow \text{Strecke gut regelbar} \\
				< 3 \rightarrow \text{Strecke schlecht regelbar}
			\end{cases}
			$
		} & $q=\frac{T_g}{T_uK_s}$ & $K_{R\pi} \qquad T_\pi=\frac{2\pi}{\omega_\pi}$
		\\ \hline
		\textbf{Regler} & \textbf{Regler\-parameter} &
		\multicolumn{2}{|p{3.5cm}|}{\textbf{Führungsverhalten} \newline $y_m$:
		Überschwingen} &
		\multicolumn{2}{|c||}{\textbf{Störverhalten}} &
		\textbf{Sprungantwort} & \textbf{Stabilitätsgrenze}
		\\ \hline
		& & kein $y_m$ & $\frac{y_m}{y_\infty} = 20 \%$ & kein $a$ & $\frac{a}{b}= 20 \%$ & &
		\\ \hline
		P		& $K_R$		& $0.3q$	& $0.7q$	& $0.3q$	& $0.7q$	& $q$		& $0.5K_{R\pi}$
		\\ \hline
		PI	& $K_R$		& $0.35q$	& $0.6q$	& $0.6q$	& $0.7q$	& $0.9q$	& $0.45K_{R\pi}$
		\\
				& $T_N$		& $1.17T_g$	& $1T_g$	& $4T_u$	& $2.33T_u$ & $3.33T_u$ &
				$0.85T_{\pi}$ \\ \hline
		PID & $K_R$		& $0.6q$	& $0.95q$	& $0.95q$	& $1.2q$	& $1.2q$	& $0.60K_{R\pi}$
		\\
			& $T_N$		& $1T_g$	& $1.36T_g$	& $2.38T_u$	& $2T_u$	& $2T_u$	& $0.50T_\pi$
		\\
			& $T_V$		& $0.5T_u$	& $0.47T_u$	& $0.42T_u$	& $0.42T_u$ & $0.5T_u$	& $0.125T_\pi$
		\\ \hline
	\end{tabular}
\end{table}

\section{Diverses}
	\subsection{Frequenzgang zweier Systeme mit Rückkopplung }
			\begin{center}
        		\includegraphics[height=3cm]{./bilder/feedback.png}
        	\end{center}

	\subsection{Graphisch Phasen-/Verstärkungsreserve}
		\includegraphics[width=7cm]{./bilder/bode-stabilitaet.png} \\
		Phasen-/Verstärkungsreserve = Phasen-/Amplitudenrand
		

\end{document}
